\documentclass[12pt, a4paper, oneside, final]{article}
\usepackage[english]{babel}
\usepackage{fontspec}
\usepackage{array, tabularx, multirow}
\usepackage[margin=2cm]{geometry}
\usepackage{graphicx}
\usepackage{xcolor}

%% Font etc.
\setmainfont{Minion 3}
\setmonofont{Fira Code}
\parindent0pt
\parsep6pt
\parskip\parsep
\frenchspacing

%% Footnotes.
\makeatletter
\renewcommand\footnoterule{%
    \kern-3\p@
    \hrule\@width.4\columnwidth
    \kern2.6\p@
}

\renewcommand\@makefntext[1]{%
    \setlength\parindent{1em}%
    \noindent
    \mbox{\llap{{}\textsuperscript{\@thefnmark}\kern.5pt}}{#1}%1
}

%% Yeet stupid American Date format.
\AtBeginDocument{
    \def\today{\the\day\space%
        \ifcase\the\month\or
            January\or February\or March\or April\or May\or June\or
            July\or August\or September\or October\or November\or December\else
        \fi\space
        \the\year
    }
}

%% Theme.
\newif\ifdarktheme \darkthemefalse

%% Colours.
\ifdarktheme
    \definecolor{icolour}{HTML}{FFCC34}
    \definecolor{rcolour}{HTML}{FF6188}
    \else
    \definecolor{icolour}{HTML}{5B3DF2}
    \definecolor{rcolour}{HTML}{C10F4A}
\fi

\def\i#1{\textcolor{icolour}{\bfseries#1}}
\def\r#1{\textcolor{rcolour}{\bfseries r#1}}

%% Data.
\title{Interpreter Manual}
\author{Ætérnal (Sirraide)}
\date{\today}

%% Document.
\makeatother
\begin{document}
\maketitle\clearpage
\tableofcontents

\clearpage\section{Introduction}
\clearpage\section{Instruction Set}
Each instruction consist of a 1-byte opcode and several operands. For the instruction encoding see §
\ref{sect:encoding}.

\subsection{\i{nop}}
This instruction unsurprisingly does nothing.

\subsection{\i{ret}}
This instruction pops the current stack frame off the frame stack and returns from the current
function. If called at the top stack frame, it will instead stop the interpreter, returning the
value of the return register.

\subsection{\i{mov} \r{d}, \r{s}/\textit{imm}}
This instruction moves an immediate or the value of \r{s} into \r{d}. \r{d} cannot be \r{0}.
\r{s}/\textit{imm} is encoded using \textit{r/imm} encoding.

\subsection{\i{add} \r{d}, \r{s1}/\textit{imm1}, \r{s2}/\textit{imm2}}
This instruction adds \r{s1}/\textit{imm1} and \r{s2}/\textit{imm2} and stores the result in \r{d}.
At most one of the two source values may be an immediate; the immediate, if any, is encoded using
\textit{r/imm} encoding. If the size of \r{d} is smaller than the sum, the result will be truncated.
Disregarding \textit{r/imm} encoding, no register may be \r{0}.

\subsection{\i{sub} \r{d}, \r{s1}/\textit{imm1}, \r{s2}/\textit{imm2}}
This instruction subtracts \r{s2}/\textit{imm2} from \r{s1}/\textit{imm1} and stores the result in
\r{d}. At most one of the two source values may be an immediate; the immediate, if any, is encoded
using \textit{r/imm} encoding. If the size of \r{d} is smaller than the difference, the result will
be truncated. Disregarding \textit{r/imm} encoding, no register may be \r{0}.

\subsection{\i{muli} \r{d}, \r{s1}/\textit{imm1}, \r{s2}/\textit{imm2}}
This instruction multiplies \r{s1}/\textit{imm1} and \r{s2}/\textit{imm2} using signed
multiplication and stores the result in \r{d}. At most one of the two source values may be an
immediate; the immediate, if any, is encoded using \textit{r/imm} encoding. If the size of \r{d} is
smaller than the product, the result will be truncated. Disregarding \textit{r/imm} encoding, no
register may be \r{0}.

This instruction is identical to \i{mulu}, save that the latter performs unsigned multiplication.

\subsection{\i{mulu} \r{d}, \r{s1}/\textit{imm1}, \r{s2}/\textit{imm2}}
This instruction multiplies \r{s1}/\textit{imm1} and \r{s2}/\textit{imm2} using unsigned
multiplication and stores the result in \r{d}. At most one of the two source values may be an
immediate; the immediate, if any, is encoded using \textit{r/imm} encoding. If the size of \r{d} is
smaller than the product, the result will be truncated. Disregarding \textit{r/imm} encoding, no
register may be \r{0}.

This instruction is identical to \i{muli}, save that the latter performs signed multiplication.

\subsection{\i{divi} \r{d}, \r{s1}/\textit{imm1}, \r{s2}/\textit{imm2}}
This instruction divides \r{s1}/\textit{imm1} by \r{s2}/\textit{imm2} using signed division and
stores the result in \r{d}. At most one of the two source values may be an immediate; the immediate,
if any, is encoded using \textit{r/imm} encoding. If the size of \r{d} is smaller than the quotient,
the result will be truncated. Disregarding \textit{r/imm} encoding, no register may be \r{0}.

This instruction is identical to \i{divu}, save that the latter performs unsigned division.

If \r{s2}/\textit{imm2}, a division error is raised.

\subsection{\i{divu} \r{d}, \r{s1}/\textit{imm1}, \r{s2}/\textit{imm2}}
This instruction divides \r{s1}/\textit{imm1} by \r{s2}/\textit{imm2} using unsigned division and
stores the result in \r{d}. At most one of the two source values may be an immediate; the immediate,
if any, is encoded using \textit{r/imm} encoding. If the size of \r{d} is smaller than the quotient,
the result will be truncated. Disregarding \textit{r/imm} encoding, no register may be \r{0}.

This instruction is identical to \i{divi}, save that the latter performs signed division.

If \r{s2}/\textit{imm2}, a division error is raised.

\subsection{\i{remi} \r{d}, \r{s1}/\textit{imm1}, \r{s2}/\textit{imm2}}
This instruction divides \r{s1}/\textit{imm1} by \r{s2}/\textit{imm2} using signed division and
stores the remainder in \r{d}. At most one of the two source values may be an immediate; the
immediate, if any, is encoded using \textit{r/imm} encoding. If the size of \r{d} is smaller than
the remainder, the result will be truncated. Disregarding \textit{r/imm} encoding, no register may
be \r{0}.

This instruction is identical to \i{remu}, save that the latter performs unsigned division.

If \r{s2}/\textit{imm2}, a division error is raised.

\subsection{\i{remu} \r{d}, \r{s1}/\textit{imm1}, \r{s2}/\textit{imm2}}
This instruction divides \r{s1}/\textit{imm1} by \r{s2}/\textit{imm2} using unsigned division and
stores the remainder in \r{d}. At most one of the two source values may be an immediate; the
immediate, if any, is encoded using \textit{r/imm} encoding. If the size of \r{d} is smaller than
the remainder, the result will be truncated. Disregarding \textit{r/imm} encoding, no register may
be \r{0}.

This instruction is identical to \i{remi}, save that the latter performs signed division.

If \r{s2}/\textit{imm2}, a division error is raised.

\subsection{\i{shl} \r{d}, \r{s1}/\textit{imm1}, \r{s2}/\textit{imm2}}
This instruction shifts \r{s1}/\textit{imm1} left by \r{s2}/\textit{imm2} and stores the result in
\r{d}. At most one of the two source values may be an immediate; the immediate, if any, is encoded
using \textit{r/imm} encoding. If the size of \r{d} is smaller than the result, the result will be
truncated. Disregarding \textit{r/imm} encoding, no register may be \r{0}.

Only the lower 6 bits of \r{s2}/\textit{imm2} are actually used.

\subsection{\i{shr} \r{d}, \r{s1}/\textit{imm1}, \r{s2}/\textit{imm2}}
This instruction shifts \r{s1}/\textit{imm1} right by \r{s2}/\textit{imm2} and stores the result in
\r{d}. At most one of the two source values may be an immediate; the immediate, if any, is encoded
using \textit{r/imm} encoding. If the size of \r{d} is smaller than the result, the result will be
truncated. Disregarding \textit{r/imm} encoding, no register may be \r{0}.

This instruction performs a logical (unsigned) shift.

Only the lower 6 bits of \r{s2}/\textit{imm2} are actually used.

\subsection{\i{sar} \r{d}, \r{s1}/\textit{imm1}, \r{s2}/\textit{imm2}}
This instruction shifts \r{s1}/\textit{imm1} right by \r{s2}/\textit{imm2} and stores the result in
\r{d}. At most one of the two source values may be an immediate; the immediate, if any, is encoded
using \textit{r/imm} encoding. If the size of \r{d} is smaller than the result, the result will be
truncated. Disregarding \textit{r/imm} encoding, no register may be \r{0}.

This instruction performs an arithmetic (signed) shift.

Only the lower 6 bits of \r{s2}/\textit{imm2} are actually used.

\subsection{\i{call} \r{a}/\textit{addr}}
This instruction calls the function at \textit{addr} or in register \r{a}. The function in
\r{a}/\textit{addr} is encoded using \textit{r/addr} encoding. For argument/return registers see §
\ref{sect:encoding}.

\subsection{\i{jmp} \r{a}/\textit{addr}}
This instruction unconditionally jumps to \textit{addr} or the address in register \r{a}. The target
address \r{a}/\textit{addr} is encoded using \textit{r/addr} encoding.

\subsection{\i{jnz} \r{c}, \r{a}/\textit{addr}}
This instruction performs a conditional jumps on the value of \r{c} to \textit{addr} or the address
in register \r{a}. The target address \r{a}/\textit{addr} is encoded using \textit{r/addr} encoding.

\subsection{\i{ld} \r{d}, \textit{imm}}
This instruction loads a value at \textit{imm} into register \r{d}. The size of \r{d} determines the
number of bytes loaded from the address.

\subsection{\i{ld} \r{d}, \r{s}, \textit{imm}}
This instruction loads a value from the address stored in \r{s} offset by \textit{imm} into register
\r{d}. The size of \r{d} determines the number of bytes loaded from the address.

\subsection{\i{st} \textit{imm}, \r{s}}
This instruction stores the value in register \r{s} into memory at the address \textit{imm}. The
size of \r{s} determines the number of bytes stored to the address.

\subsection{\i{st} \r{d}, \textit{imm}, \r{s}}
This instruction stores the value in register \r{s} to the address in \r{d} offset by \textit{imm}.
The size of \r{s} determines the number of bytes loaded from the address.

\clearpage\section{Encoding}\label{sect:encoding}

\end{document}

